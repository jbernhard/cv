\documentclass[letterpaper,10pt]{article}

\usepackage[utf8]{inputenc}
\usepackage[T1]{fontenc}
\usepackage{lmodern}

\usepackage[margin=1.5in]{geometry}

\usepackage{xcolor}
\definecolor{theblue}{RGB}{22,79,149}
\usepackage[colorlinks=true,urlcolor=theblue]{hyperref}

\usepackage{titlesec}
\titleformat{\section}[hang]{\sc\Large}{}{0pt}{}
\titlespacing{\section}{0pt}{1.75em}{.5em}
\titleformat{\subsection}[runin]{\bf}{}{0pt}{}
\titlespacing{\subsection}{0pt}{1ex}{0pt}

\usepackage{enumitem}
\setlist[itemize]{itemsep=.75ex,parsep=0pt,topsep=.75ex,leftmargin=1.25em,label=\raisebox{.35ex}{\color{theblue}\tiny\textbullet}}
\setlist[enumerate]{itemsep=.25ex,parsep=.5ex,topsep=.75ex,leftmargin=*}

\setlength{\parindent}{0pt}

\newcommand{\entry}[3]{\vspace{.5em plus .1em minus .1em}\textbf{#1}, #2 \hfill #3}

\usepackage[style=phys,eprint=true,url=true]{biblatex}
\defbibenvironment{bibliography}{\vspace{.5em}\begin{enumerate}[itemsep=.8ex]}{\end{enumerate}}{\item}
\addbibresource{papers.bib}


\begin{document}


\begin{center}
  \Large
  Curriculum Vitae:  Jonah Emery Bernhard \\[1ex]
  \normalsize\rm
  \url{https://jbernhard.xyz} \\
  \href{mailto:jonah.bernhard@gmail.com}{\nolinkurl{jonah.bernhard@gmail.com}} \\
  \href{https://github.com/jbernhard}{\nolinkurl{github/jbernhard}}
\end{center}


\section{Education}

\entry{Duke University}{Durham, North Carolina}{August 2011--March 2018}

\begin{itemize}
  \item Ph.D. Physics, defended March 30, 2018
  \item Dissertation: Bayesian parameter estimation for relativistic heavy-ion collisions
\end{itemize}

\entry{Swarthmore College}{Swarthmore, Pennsylvania}{September 2007--May 2011}

\begin{itemize}
  \item B.A.\ Chemical Physics with Highest Honors, 2011
  \item Senior thesis: Lorentz violation in solar-neutrino oscillations \\
    Nominated for APS Apker Award
\end{itemize}


\section{Research Experience}

\entry{Graduate research assistant}{Duke University}{January 2012--Present}

\begin{itemize}
  \item Quantitatively estimated fundamental properties of the quark-gluon plasma created in ultra-relativistic heavy-ion collisions by applying Bayesian statistical methodology.
  \item Developed computational models of heavy-ion collisions;
    ran large-scale simulations on high-performance computational systems.
  \item Based Ph.D.\ dissertation on this work.
  \item Advisor: Steffen A.\ Bass
\end{itemize}

\entry{Undergraduate research assistant}{Swarthmore College}{January 2010--May 2011}

\begin{itemize}
  \item Modeled and quantified the possible effects of Lorentz violation on solar neutrinos.
  \item Based undergraduate thesis on this work.
  \item Advisor: Matthew Mewes
\end{itemize}

\entry{Summer research fellow}{Swarthmore College}{June--August 2009}

\begin{itemize}
  \item Performed experimental studies of the excited states of nitric oxide gas.
  \item Advisor: Thomas Stephenson
\end{itemize}

\pagebreak


\nocite{*}
{\raggedright\printbibliography[heading=bibintoc, title={Publications}]}


\section{Presentations}

\begin{enumerate}
  \item \entry{Quark Matter 2017}{Chicago, Illinois}{February 7, 2017} \par
    Characterization of the initial state and QGP medium from a \\
    combined Bayesian analysis of LHC data at 2.76 and 5.02 TeV

  \item \entry{Institute for Nuclear Theory workshop}{University of Washington}{June 15, 2016} \par
    Precision extraction of QGP properties with quantified uncertainties \\
    Part II: methodology and results

  \item \entry{Quark Matter 2015}{Kobe, Japan}{September 29, 2015} \par
    Poster: Bayesian characterization of the \\
    initial state and QGP medium

  \item \entry{Institute for Nuclear Theory workshop}{University of Washington}{July 14, 2015} \par
    Quantifying properties of hot and dense QCD matter \\
    through systematic model-to-data comparison

  \item \entry{2015 RHIC \& AGS users' meeting}{Brookhaven National Lab}{June 9, 2015} \par
    Bayesian methods for constraining initial conditions and viscosity

  \item \entry{CIPANP 2015}{Vail, Colorado}{May 22, 2015} \par
    Quantifying properties of hot and dense QCD matter \\
    through systematic model-to-data comparison

  \item \entry{MADAI workshop}{Michigan State University}{July 7, 2014} \par
    Model-to-data comparison for event-by-event flow distributions: \\
    progress and pitfalls

  \item \entry{Quark Matter 2014}{Darmstadt, Germany}{May 20, 2014} \par
    Poster: QGP parameter extraction via a global \\
    analysis of event-by-event flow coefficient distributions

  \item \entry{ECT*}{Trento, Italy}{May 13, 2014} \par
    QGP parameter extraction via a global \\
    analysis of event-by-event flow coefficient distributions

  \item \entry{JET bulk group meeting}{Duke University}{March 24, 2014} \par
    QGP parameter extraction via a global analysis of \\
    event-by-event flow coefficient distributions

  \item \entry{DNP fall 2013 meeting}{Newport News, Virginia}{October 26, 2013} \par
    QGP parameter extraction via a global analysis of \\
    event-by-event flow coefficient distributions

  \item \entry{Sampling workshop}{Frankfurt, Germany}{July 20, 2013} \par
    QGP parameter extraction via a global analysis of \\
    event-by-event flow coefficient distributions

  \item \entry{Senior honors talk}{Swarthmore College}{April 27, 2011} \par
    Lorentz violation in solar-neutrino oscillations

  \item \entry{CPT '10 meeting}{Indiana University}{July 1, 2010} \par
    Poster: Lorentz violation in solar-neutrino oscillations
\end{enumerate}


\section{Additional Academic Training}

\entry{JET Summer School 2014}{University of California, Davis}{June 19--21, 2014}

\entry{ECT* Doctoral Training Program}{Trento, Italy}{April 7--May 16, 2014}

\entry{JET Summer School 2013}{Ohio State University}{June 12--14, 2013}


\section{Teaching Experience}

\entry{Recitation instructor}{Duke University}{August 2011--April 2013}

\begin{itemize}
  \item Introductory Mechanics
  \item Introductory Electricity \& Magnetism
\end{itemize}

\entry{Teaching assistant}{Swarthmore College}{September 2009--April 2011}

\begin{itemize}
  \item Introductory Mechanics
  \item Introductory Electricity \& Magnetism
  \item Mathematical Methods of Physics
  \item Organic Chemistry
\end{itemize}

\end{document}
